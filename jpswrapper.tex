%% This package is distributed under the terms of the GPLv3 or later, or the
%% LPPL 1.3c or later, which ever license fits your needs the best.
%%
%% Copyright (C) 2017-2018 by Jonathan P. Spratte
\documentclass[a4paper,10pt]{scrartcl}

\usepackage[utf8]{inputenc}
\usepackage[T1]{fontenc}
\usepackage{lmodern}
\usepackage{booktabs}
\usepackage{blindtext}
\usepackage{microtype}
\usepackage[all]{jpswrapper}
\usepackage{multicol}
\usepackage[colorlinks]{hyperref}
\JPSWrapperOptions[wrapfig]{alternate}

\newenvironment{codedescription}{%
  \parindent=-3em
  \parskip=1em plus 2pt minus 2pt
  \par
}{\par}
\newenvironment{options}{%
  \renewcommand*{\descriptionlabel}[1]{\hspace{\labelsep}\texttt{##1}}%
  \begin{description}%
}{%
  \end{description}%
}
\makeatletter
\newenvironment{codedescription*}{%
  \parindent=-3em
  \parskip=1em plus 2pt minus 2pt
  \vskip-\lastskip
  \hspace*{\parindent}\ignorespaces
}{\par}
\newcommand*\dependencies[2][3]{%
  \@ifstar
    {\section{Dependencies}}
    {\subsubsection{Additional Dependencies}}%
  \begin{multicols}{#1}
    \begin{itemize}%
      \@for\cs:={#2}\do{%
        \item\texttt{\cs}}%
    \end{itemize}%
  \end{multicols}}%
\makeatother

\newcommand*\optionTF{\textit{true}{\normalfont|}\textit{false}}

\title{The \texttt{jpswrapper} package}
\author{Jonathan P. Spratte}
\makeatletter
\date{\jpswrapper@version\ \ \jpswrapper@date}
\makeatother

\begin{document}
\maketitle
\tableofcontents
\section{Macros}
The package provides a central macro to switch some options of the modules:
\begin{codedescription}
\texttt{\string\JPSWrapperOptions[\textit{module}]\{\textit{options}\}}\\
  The options for each module can't be used as options to \verb|\usepackage| as
  they are available only after the package was loaded. This command can be used
  to specify options to the package without the optional argument, or for a
  single module with the optional argument. Alternatively you could specify
  module options with
  \texttt{\string\JPSWrapperOptions\{\textit{module}\_module/\textit{option}\}}.
\end{codedescription}

\section{Modules}
This section contains the modules of \texttt{jpswrapper}. Each module should
correspond to one package for which at least one wrapper is created. If you
load one module the corresponding package gets loaded by
\texttt{\string\RequirePackage\{\textit{module}\}}. In addition modules can have
additional packages as dependencies.

For each module there is a corresponding boolean option to this package which
specifies whether to load that module or not. If thoseAdditional options are:
\begin{codedescription}
\texttt{all=\optionTF}\\
  if set \texttt{true} every module gets loaded (might be overridden by
  \texttt{except}, \texttt{all=false} or \texttt{\textit{module}=false}). No
  value is considered as \texttt{true}.

\texttt{except}\\
  specifies which packages shouldn't be loaded. Accepts a comma separated list.
\end{codedescription}

\subsection{\texttt{wrapfig}}
This module creates a wrapper around the \texttt{wrapfloat} environment of the
\texttt{wrapfig} package. It doesn't change anything inside \texttt{wrapfig}'s
code so the stuff contained in this module and the package it should wrap can
coexist.

\subsubsection{Environments}
The module provides a new environment which is meant to give an easier and
clearer interface to \texttt{wrapfig}'s environment \texttt{wrapfloat}.

\begin{codedescription}
\texttt{\string\begin\{JPSwrapfloat\}[\textit{options}]}\\
  Sets a \texttt{wrapfloat} environment. It doesn't require any more options and
  has no mandatory arguments. The following options are available for this
  environment:
  \begin{options}
    \item[type=\textit{type}]\mbox{}\\
      the type of float (\texttt{table} or \texttt{figure}). Default is
      \texttt{table}
    \item[table]\mbox{}\\
      same as \texttt{type=table}
    \item[fig\textit{ure}]\mbox{}\\
      same as \texttt{type=figure}. \texttt{fig} is a short form
    \item[hang=\textit{dim}]\mbox{}\\
      the overhang into the margin
    \item[lines=\textit{num}]\mbox{}\\
      number of lines if you want to specify them
    \item[width=\textit{dim}]\mbox{}\\
      the width, if you want to specify it. Default is \texttt{0pt} which
      results in \texttt{wrapfloat} calculating it.
    \item[\textit{placement}]\mbox{}\\
      the placement. Accepted are the same as for \texttt{wrapfig} (\texttt{R},
      \texttt{r}, \texttt{L}, \texttt{l}, \texttt{O}, \texttt{o}, \texttt{I},
      \texttt{i})
    \item[*]\mbox{}\\
      tries to implement another automatic determination of the width. This
      \emph{might} work better than \texttt{wrapfig}'s method, but is based on
      the same idea. We redefine the caption command to do nothing during the
      width determination and hope nothing else is contained inside the
      environment which isn't allowed to be placed inside a \verb|\hbox|. With
      this wrapper, the redefinition is done with the macro
      \verb|\JPSwrapfigRedefineCaption|. It is applied only when the option
      \texttt{width=0pt} is used (or not used if you didn't change anything on
      the default \texttt{width}) . The test is done inside a group, you still
      have to watch out if you use anything changing a counter or the like
      globally.
  \end{options}
\end{codedescription}

\subsubsection{Module Options}
\begin{codedescription*}
\texttt{alternate=\optionTF}\\
  If this option is set, the corresponding opposite of the placement is used for
  the next placement (e.\,g., if you use \texttt{r} next time \texttt{l} is
  used). The resulting placement gets overridden by options given directly to
  \texttt{JPSwrapfloat} and by those used as user defaults (with
  \verb|\JPSwrapfloatUserDefaults|). If it isn't set, the same placement is used
  again.
\end{codedescription*}

\subsubsection{Additional Commands}
\begin{codedescription*}
\texttt{\string\JPSwrapfloatUserDefaults\textit{*}\{\textit{options}\}}\\
  Sets up a list of options which is processed prior to the options you provide
  to the \texttt{JPSwrapfloat} environment. If the optional star is given, the
  list is cleared prior to adding the specified \texttt{\textit{options}}.

\verb|\JPSwrapfigRedefineCaption|\\
  This is the macro which redefines \verb|\caption| (and if defined
  \verb|\captionof|) for the automatic width description if a \texttt{*} was one
  of the options given to \texttt{JPSwrapfloat}. Change this if you use anything
  which makes \verb|\caption| accept more than just one optional and one
  mandatory argument. You might also including other stuff you have to switch
  off during width determination.
\end{codedescription*}

\dependencies{environ,etoolbox,wrapfig}

\section{Dependencies}
The package requires the packages \texttt{xparse} and \texttt{l3keys2e}.

\section{License}
This package is distributed under the terms of the GPLv3 or later, or the
LPPL~1.3c or later, which ever license fits your needs the best.

\end{document}
